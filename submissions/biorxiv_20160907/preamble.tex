\documentclass[10pt,letterpaper]{article}
\usepackage[top=0.85in,left=2.75in,footskip=0.75in,marginparwidth=2in]{geometry}

% use Unicode characters - try changing the option if you run into troubles with special characters (e.g. umlauts)
\usepackage[utf8]{inputenc}

% clean citations
\usepackage{cite}

% hyperref makes references clicky. use \url{www.example.com} or \href{www.example.com}{description} to add a clicky url
\usepackage{nameref,hyperref}

% line numbers
\usepackage[right]{lineno}

% improves typesetting in LaTeX
\usepackage{microtype}
\DisableLigatures[f]{encoding = *, family = * }

% text layout - change as needed
\raggedright
\setlength{\parindent}{0.5cm}
\textwidth 5.25in 
\textheight 8.75in

% Remove % for double line spacing
%\usepackage{setspace} 
%\doublespacing

% use adjustwidth environment to exceed text width (see examples in text)
\usepackage{changepage}

% adjust caption style
\usepackage[aboveskip=1pt,labelfont=bf,labelsep=period,singlelinecheck=off]{caption}

% remove brackets from references
\makeatletter
\renewcommand{\@biblabel}[1]{\quad#1.}
\makeatother

% headrule, footrule and page numbers
\usepackage{lastpage,fancyhdr,graphicx}
\usepackage{epstopdf}
\pagestyle{myheadings}
\pagestyle{fancy}
\fancyhf{}
\rfoot{\thepage/\pageref{LastPage}}
\renewcommand{\footrule}{\hrule height 2pt \vspace{2mm}}
\fancyheadoffset[L]{2.25in}
\fancyfootoffset[L]{2.25in}

% use \textcolor{color}{text} for colored text (e.g. highlight to-do areas)
\usepackage{color}

% define custom colors (this one is for figure captions)
\definecolor{Gray}{gray}{.25}

% this is required to include graphics
\usepackage{graphicx}

% use if you want to put caption to the side of the figure - see example in text
\usepackage{sidecap}

% use for have text wrap around figures
\usepackage{wrapfig}
\usepackage[pscoord]{eso-pic}
\usepackage[fulladjust]{marginnote}
\reversemarginpar

% Adding multirow.
\usepackage{multirow}

% document begins here
\begin{document}
\vspace*{0.35in}

% title goes here:
\begin{flushleft}
{\Large
\textbf\newline{Overcoming confounding plate effects in differential expression analyses of single-cell RNA-seq data}
}
\newline
% authors go here:
\\
Aaron T. L. Lun\textsuperscript{1,*},
John C. Marioni\textsuperscript{1,2,3}
\\
\bigskip
\bf{1} Cancer Research UK Cambridge Institute, University of Cambridge, Li Ka Shing Centre, Robinson Way, Cambridge CB2 0RE, United Kingdom
\\
\bf{2} EMBL European Bioinformatics Institute, Wellcome Genome Campus, Hinxton, Cambridge CB10 1SD, United Kingdom
\\
\bf{3} Wellcome Trust Sanger Institute, Wellcome Genome Campus, Hinxton, Cambridge CB10 1SA, United Kingdom
\\
\bigskip
* aaron.lun@cruk.cam.ac.uk

\end{flushleft}

\newcommand{\revised}[1]{#1}

\section*{Abstract}
An increasing number of studies are using single-cell RNA-sequencing (scRNA-seq) to characterize the gene expression profiles of individual cells.
One common analysis applied to scRNA-seq data involves detecting differentially expressed (DE) genes between cells in different biological groups.
However, many experiments are designed such that the cells to be compared are processed in separate plates or chips, 
    meaning that the groupings are confounded with systematic plate effects.
This confounding aspect is frequently ignored in DE analyses of scRNA-seq data.
In this article, we demonstrate that failing to consider plate effects in the statistical model results in loss of type I error control.
A solution is proposed whereby counts are summed from all cells in each plate and the count sums for all plates are used in the DE analysis.
This restores type I error control in the presence of plate effects without compromising detection power in simulated data.
Summation is also robust to varying numbers and library sizes of cells on each plate.
Similar results are observed in DE analyses of real data where the use of count sums instead of single-cell counts \revised{improves specificity and the ranking of relevant genes}.
This suggests that summation can assist in maintaining statistical rigour in DE analyses of scRNA-seq data with plate effects.

% now start line numbers
\linenumbers

